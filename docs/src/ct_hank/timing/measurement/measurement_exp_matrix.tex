\documentclass[12 pt, oneside]{article}
\textheight 9 in
\textwidth 6.5 in
\topmargin 0 in
\oddsidemargin .3 in
\evensidemargin .3 in
\usepackage{amssymb}
\usepackage{amsmath,amsthm}
\usepackage{amsbsy,paralist}
\usepackage{appendix}
\usepackage{natbib}
\usepackage{amsfonts}
\usepackage{graphicx}
\usepackage{epsfig}
\usepackage{color}
\usepackage{mathrsfs}
\usepackage{fancyhdr}
\usepackage{setspace}
%\usepackage[nodisplayskipstretch]{setspace}
\usepackage{fullpage}
\usepackage{cancel}
\usepackage{lipsum}
\usepackage{subfig}
\let\oldemptyset\emptyset
\let\emptyset\varnothing
\setlength{\parindent}{1cm}
\newtheorem*{thm}{Theorem}
\newtheorem*{lem}{Lemma}
\newtheorem{lemma}{Lemma}[section]
\newtheorem{lemfull}{Lemma}[section]
\newtheorem*{cor}{Corollary}
\newtheorem{corollary}{Corollary}[section]
\newtheorem{theorem}{Theorem}[section]
\newtheorem{thmfull}{Theorem}[section]
\newtheorem*{prop}{Proposition}
\newtheorem{proposition}{Proposition}[section]
\newtheorem{propfull}{Proposition}[section]
\theoremstyle{definition}
\newtheorem*{remark}{Remarks}
\theoremstyle{definition}
\newtheorem*{eg}{Example}
\theoremstyle{definition}
\newtheorem*{defn}{Definition}
\newtheorem{definition}{Definition}[section]
\newcommand{\bigsum}[2]{\sum\limits_{#1}^{#2}}
\newcommand{\bigprod}[2]{\prod\limits_{#1}^{#2}}
\newcommand{\nlim}{\lim_{n\ra\infty}}
\newcommand{\vecx}{\vec{x}}
\newcommand{\vecy}{\vec{y}}
\DeclareMathOperator{\Char}{Char}
\DeclareMathOperator{\orb}{orb}
\DeclareMathOperator{\stab}{stab}
\DeclareMathOperator{\Aut}{Aut}
\DeclareMathOperator{\Inn}{Inn}
\DeclareMathOperator{\lcm}{lcm}
\DeclareMathOperator{\card}{card}
\DeclareMathOperator{\Cl}{Cl}
\DeclareMathOperator{\Int}{Int}
\DeclareMathOperator{\var}{Var}
\DeclareMathOperator{\io}{i.o.}
\DeclareMathOperator{\sgn}{sgn}
\DeclareMathOperator{\tr}{trace}
\newcommand{\bfc}{\mathbf{c}}
\newcommand{\bfb}{\mathbf{b}}
\newcommand{\bfa}{\mathbf{a}}
\newcommand{\bfd}{\mathbf{d}}
\newcommand{\bff}{\mathbf{f}}
\newcommand{\bfh}{\mathbf{h}}
\newcommand{\bfg}{\mathbf{g}}
\newcommand{\bfi}{\mathbf{i}}
\newcommand{\bfj}{\mathbf{j}}
\newcommand{\bfk}{\mathbf{k}}
\newcommand{\bfl}{\mathbf{l}}
\newcommand{\bfm}{\mathbf{m}}
\newcommand{\bfn}{\mathbf{n}}
\newcommand{\bfo}{\mathbf{o}}
\newcommand{\bfp}{\mathbf{p}}
\newcommand{\bfq}{\mathbf{q}}
\newcommand{\bfr}{\mathbf{r}}
\newcommand{\bfs}{\mathbf{s}}
\newcommand{\bft}{\mathbf{t}}
\newcommand{\bfx}{\mathbf{x}}
\newcommand{\bfy}{\mathbf{y}}
\newcommand{\bfz}{\mathbf{z}}
\newcommand{\bfu}{\mathbf{u}}
\newcommand{\bfv}{\mathbf{v}}
\newcommand{\bfw}{\mathbf{w}}
\newcommand{\bfX}{\mathbf{X}}
\newcommand{\bfY}{\mathbf{Y}}
\newcommand{\bfA}{\mathbf{A}}
\newcommand{\bfB}{\mathbf{B}}
\newcommand{\bfC}{\mathbf{C}}
\newcommand{\bfD}{\mathbf{D}}
\newcommand{\bfE}{\mathbf{E}}
\newcommand{\bfF}{\mathbf{F}}
\newcommand{\bfG}{\mathbf{G}}
\newcommand{\bfH}{\mathbf{H}}
\newcommand{\bfI}{\mathbf{I}}
\newcommand{\bfJ}{\mathbf{J}}
\newcommand{\bfK}{\mathbf{K}}
\newcommand{\bfL}{\mathbf{L}}
\newcommand{\bfU}{\mathbf{U}}
\newcommand{\bfV}{\mathbf{V}}
\newcommand{\bfW}{\mathbf{W}}
\newcommand{\bfZ}{\mathbf{Z}}
\newcommand{\bfM}{\mathbf{M}}
\newcommand{\bfN}{\mathbf{N}}
\newcommand{\bfQ}{\mathbf{Q}}
\newcommand{\bfO}{\mathbf{O}}
\newcommand{\bfR}{\mathbf{R}}
\newcommand{\bfS}{\mathbf{S}}
\newcommand{\bfT}{\mathbf{T}}
\newcommand{\bfP}{\mathbf{P}}
\newcommand{\bfzero}{\mathbf{0}}
\newcommand{\R}{\mathbb{R}}
\newcommand{\E}{\mathbb{E}}
\newcommand{\N}{\mathbb{N}}
\newcommand{\Q}{\mathbb{Q}}
\newcommand{\Z}{\mathbb{Z}}
\newcommand{\curA}{\mathscr{A}}
\newcommand{\curC}{\mathscr{C}}
\newcommand{\curD}{\mathscr{D}}
\newcommand{\curE}{\mathscr{E}}
\newcommand{\curH}{\mathscr{H}}
\newcommand{\curI}{\mathscr{I}}
\newcommand{\curJ}{\mathscr{J}}
\newcommand{\curK}{\mathscr{K}}
\newcommand{\curN}{\mathscr{N}}
\newcommand{\curO}{\mathscr{O}}
\newcommand{\curQ}{\mathscr{Q}}
\newcommand{\curS}{\mathscr{S}}
\newcommand{\curT}{\mathscr{T}}
\newcommand{\curU}{\mathscr{U}}
\newcommand{\curV}{\mathscr{V}}
\newcommand{\curW}{\mathscr{W}}
\newcommand{\curZ}{\mathscr{Z}}
\newcommand{\curB}{\mathscr{B}}
\newcommand{\curF}{\mathscr{F}}
\newcommand{\curG}{\mathscr{G}}
\newcommand{\curM}{\mathscr{M}}
\newcommand{\curL}{\mathscr{L}}
\newcommand{\curP}{\mathscr{P}}
\newcommand{\curR}{\mathscr{R}}
\newcommand{\curX}{\mathscr{X}}
\newcommand{\curY}{\mathscr{Y}}
\newcommand{\calA}{\mathcal{A}}
\newcommand{\calD}{\mathcal{D}}
\newcommand{\RA}{\Rightarrow}
\newcommand{\ra}{\rightarrow}
\newcommand{\fd}{\vspace{2.5mm}}
\newcommand{\as}{\vspace{1mm}}
\begin{document}
Suppose we have the continuous-time transition equation
\[dX_t = TX_t\,dt + R\, dW_t,  \]
where $X_t\in \R^n$, $T\in \R^{n\times n}$, $R\in \R^{n\times m}$, and $dW_t$ is a $m$-dimensional standard Brownian motion. In expectation, we have that this system behaves according to
\[ dX_t = TX_t\,dt, \]
which corresponds to the linear ODE system\footnote{Formally, any stochastic differential equation (SDE) is an \textit{integral} equation. In our case, our SDE is actually short hand for
\[X_t - X_0 = \int_0^t TX_s\, ds + \int_0^t R\, dW_s.  \]
The second integral is an Ito integral, which is a martingale. With standard Brownian motion, it has mean zero, so in expectation, our equation becomes
\[\E[X_t - X_0] = \int_0^t TX_s\,ds,  \]
which is the \textit{integral} equivalent of a linear ODE (i.e. every ODE can be represented with integrals).
}
\[ \dfrac{dX_t}{dt} = TX_t, \]
yielding the standard solution
\[ X_t = \exp((t-\tau)T)X_\tau, \]
where $X_\tau$ is some initial condition (though $\tau$ need not be zero).

Our problem is to find a matrix $Z$ which convert flows into stocks. Our states, $X_t$, represents the instantaneous state variable, so, for example, if output is a state, then $X_t$ tracks the \textit{rate} of output. Therefore, to acquire the total output procued, a \textit{stock} variable, we must integrate the path of $X_t$. In the case that $T$ is invertible, we have the analytical solution\footnote{To develop intuition for this, it may be useful to study the scalar case, where the invertibility of $T$ does not matter, as it will just be some real.}
\[\left| \int_\tau^S X_t\, dt\right| = \left|T^{-1}\exp((t-\tau)T)X_\tau - T^{-1}\right|. \]
Note that we use absolute values here to allow for $\tau > S$ in the case that we want to determine output by integrating backwards in time (i.e. I observe my current state and guess the stock of output by extrapolating backward rather than observing my past state and guessing its evolution forward from there).

However, $T$ is not always invertible, and in test trials with HANK models, $T$ won't be invertible. Therefore, we approximate it using the definition of exponential matrices. Given any matrix $T$, we have that
\[ \exp((t-\tau) T) = \sum_{n = 0}^\infty \dfrac{((t-\tau)T)^n}{n!}, \]
where $((t-\tau)T)^0 = I$. We can integrate term by term by the dominated convergence theorem\footnote{Since we know the exponential matrix converges, we can bound the tail above by some constant and integrate the resulting \textit{finite} sum of powers of $T$, plus some constant. Since this function is integrable and dominates $\exp((t-\tau)T)$, we can apply the Lebesgue dominated convergence theorem and integrate term by term.}, yielding
\begin{align*}
& \int_{\tau}^S \exp((t-\tau)T)\, dt = \sum_{ n = 0}^\infty \int_\tau^S \dfrac{((t-\tau)T)^n}{n!}\\
& = (S-\tau) +\dfrac{(S-\tau)^2 T}{2!} + \dfrac{(S-\tau)^3T^2}{3!} + \dfrac{(S-\tau)^4T^3}{4!} + \cdots\\
& = (S-\tau)\sum_{n = 0}^\infty \dfrac{((S-\tau)T)^n}{n!}.
\end{align*}
We can therefore write
\begin{align*}
\int_\tau^S \exp((t-\tau)T) X_\tau\, ds & = \left(\int_\tau^S \exp((t-\tau)T)\, ds\right) X_\tau= \left((S-\tau)\sum_{n = 0}^\infty \dfrac{((S-\tau)T)^n}{n!}\right)X_\tau.
\end{align*}





\end{document}
