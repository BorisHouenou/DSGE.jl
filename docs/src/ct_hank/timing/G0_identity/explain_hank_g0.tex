\documentclass[12 pt, oneside]{article}
\textheight 9 in
\textwidth 6.5 in
\topmargin 0 in
\oddsidemargin .3 in
\evensidemargin .3 in
\usepackage{amssymb}
\usepackage{amsmath,amsthm}
\usepackage{amsbsy,paralist}
\usepackage{appendix}
\usepackage{natbib}
\usepackage{amsfonts}
\usepackage{graphicx}
\usepackage{epsfig}
\usepackage{color}
\usepackage{mathrsfs}
\usepackage{fancyhdr}
\usepackage{setspace}
%\usepackage[nodisplayskipstretch]{setspace}
\usepackage{fullpage}
\usepackage{cancel}
\usepackage{lipsum}
\usepackage{subfig}
\let\oldemptyset\emptyset
\let\emptyset\varnothing
\setlength{\parindent}{1cm}
\newtheorem*{thm}{Theorem}
\newtheorem*{lem}{Lemma}
\newtheorem{lemma}{Lemma}[section]
\newtheorem{lemfull}{Lemma}[section]
\newtheorem*{cor}{Corollary}
\newtheorem{corollary}{Corollary}[section]
\newtheorem{theorem}{Theorem}[section]
\newtheorem{thmfull}{Theorem}[section]
\newtheorem*{prop}{Proposition}
\newtheorem{proposition}{Proposition}[section]
\newtheorem{propfull}{Proposition}[section]
\theoremstyle{definition}
\newtheorem*{remark}{Remarks}
\theoremstyle{definition}
\newtheorem*{eg}{Example}
\theoremstyle{definition}
\newtheorem*{defn}{Definition}
\newtheorem{definition}{Definition}[section]
\newcommand{\bigsum}[2]{\sum\limits_{#1}^{#2}}
\newcommand{\bigprod}[2]{\prod\limits_{#1}^{#2}}
\newcommand{\nlim}{\lim_{n\ra\infty}}
\newcommand{\vecx}{\vec{x}}
\newcommand{\vecy}{\vec{y}}
\DeclareMathOperator{\Char}{Char}
\DeclareMathOperator{\orb}{orb}
\DeclareMathOperator{\stab}{stab}
\DeclareMathOperator{\Aut}{Aut}
\DeclareMathOperator{\Inn}{Inn}
\DeclareMathOperator{\lcm}{lcm}
\DeclareMathOperator{\card}{card}
\DeclareMathOperator{\Cl}{Cl}
\DeclareMathOperator{\Int}{Int}
\DeclareMathOperator{\var}{Var}
\DeclareMathOperator{\io}{i.o.}
\DeclareMathOperator{\sgn}{sgn}
\DeclareMathOperator{\tr}{trace}
\newcommand{\bfc}{\mathbf{c}}
\newcommand{\bfb}{\mathbf{b}}
\newcommand{\bfa}{\mathbf{a}}
\newcommand{\bfd}{\mathbf{d}}
\newcommand{\bff}{\mathbf{f}}
\newcommand{\bfh}{\mathbf{h}}
\newcommand{\bfg}{\mathbf{g}}
\newcommand{\bfi}{\mathbf{i}}
\newcommand{\bfj}{\mathbf{j}}
\newcommand{\bfk}{\mathbf{k}}
\newcommand{\bfl}{\mathbf{l}}
\newcommand{\bfm}{\mathbf{m}}
\newcommand{\bfn}{\mathbf{n}}
\newcommand{\bfo}{\mathbf{o}}
\newcommand{\bfp}{\mathbf{p}}
\newcommand{\bfq}{\mathbf{q}}
\newcommand{\bfr}{\mathbf{r}}
\newcommand{\bfs}{\mathbf{s}}
\newcommand{\bft}{\mathbf{t}}
\newcommand{\bfx}{\mathbf{x}}
\newcommand{\bfy}{\mathbf{y}}
\newcommand{\bfz}{\mathbf{z}}
\newcommand{\bfu}{\mathbf{u}}
\newcommand{\bfv}{\mathbf{v}}
\newcommand{\bfw}{\mathbf{w}}
\newcommand{\bfX}{\mathbf{X}}
\newcommand{\bfY}{\mathbf{Y}}
\newcommand{\bfA}{\mathbf{A}}
\newcommand{\bfB}{\mathbf{B}}
\newcommand{\bfC}{\mathbf{C}}
\newcommand{\bfD}{\mathbf{D}}
\newcommand{\bfE}{\mathbf{E}}
\newcommand{\bfF}{\mathbf{F}}
\newcommand{\bfG}{\mathbf{G}}
\newcommand{\bfH}{\mathbf{H}}
\newcommand{\bfI}{\mathbf{I}}
\newcommand{\bfJ}{\mathbf{J}}
\newcommand{\bfK}{\mathbf{K}}
\newcommand{\bfL}{\mathbf{L}}
\newcommand{\bfU}{\mathbf{U}}
\newcommand{\bfV}{\mathbf{V}}
\newcommand{\bfW}{\mathbf{W}}
\newcommand{\bfZ}{\mathbf{Z}}
\newcommand{\bfM}{\mathbf{M}}
\newcommand{\bfN}{\mathbf{N}}
\newcommand{\bfQ}{\mathbf{Q}}
\newcommand{\bfO}{\mathbf{O}}
\newcommand{\bfR}{\mathbf{R}}
\newcommand{\bfS}{\mathbf{S}}
\newcommand{\bfT}{\mathbf{T}}
\newcommand{\bfP}{\mathbf{P}}
\newcommand{\bfzero}{\mathbf{0}}
\newcommand{\R}{\mathbb{R}}
\newcommand{\E}{\mathbb{E}}
\newcommand{\N}{\mathbb{N}}
\newcommand{\Q}{\mathbb{Q}}
\newcommand{\Z}{\mathbb{Z}}
\newcommand{\curA}{\mathscr{A}}
\newcommand{\curC}{\mathscr{C}}
\newcommand{\curD}{\mathscr{D}}
\newcommand{\curE}{\mathscr{E}}
\newcommand{\curH}{\mathscr{H}}
\newcommand{\curI}{\mathscr{I}}
\newcommand{\curJ}{\mathscr{J}}
\newcommand{\curK}{\mathscr{K}}
\newcommand{\curN}{\mathscr{N}}
\newcommand{\curO}{\mathscr{O}}
\newcommand{\curQ}{\mathscr{Q}}
\newcommand{\curS}{\mathscr{S}}
\newcommand{\curT}{\mathscr{T}}
\newcommand{\curU}{\mathscr{U}}
\newcommand{\curV}{\mathscr{V}}
\newcommand{\curW}{\mathscr{W}}
\newcommand{\curZ}{\mathscr{Z}}
\newcommand{\curB}{\mathscr{B}}
\newcommand{\curF}{\mathscr{F}}
\newcommand{\curG}{\mathscr{G}}
\newcommand{\curM}{\mathscr{M}}
\newcommand{\curL}{\mathscr{L}}
\newcommand{\curP}{\mathscr{P}}
\newcommand{\curR}{\mathscr{R}}
\newcommand{\curX}{\mathscr{X}}
\newcommand{\curY}{\mathscr{Y}}
\newcommand{\calA}{\mathcal{A}}
\newcommand{\calD}{\mathcal{D}}
\newcommand{\RA}{\Rightarrow}
\newcommand{\ra}{\rightarrow}
\newcommand{\fd}{\vspace{2.5mm}}
\newcommand{\as}{\vspace{1mm}}
\begin{document}
Without any aggregate shocks and after applying Ito's lemma, the steady state in a HANK model can be represented as the system
\begin{align*}
\rho v_t & = \max_c u(c) +\dfrac{1}{dt} \E_t[\calA v_t],\\
\dfrac{dg_t}{dt} & = -\calA^*g_t,\\
p_t & = F(g_t),
\end{align*}
where $\calA$ is the generator of an Ito diffusion (or jump diffusion or semi-martingale), $\calA^*$ is the adjoint, etc.

In the steady state, where we set aggregate shocks to zero, we can represent the discretized system as
\begin{align*}
\rho \bfv_t & = \bfu(\bfv_t) + \bfA(\bfv_t;\bfp_t) \bfv_t\\
\bfzero & = \bfA(\bfv_t;\bfp_t)^T\bfg_t\\
\bfp_t & = \bfF(\bfg_t).
\end{align*}
In this system, $\bfv, \bfg, \bfp$ are all now vectors at points in a discrete grid, with $\bfA$ performing a variety of finite difference operations. To arrive at the system with aggregate shocks $Z_t$, we have
\begin{align*}
\rho \bfv_t & = \bfu(\bfv_t) + \bfA(\bfv_t;\bfp_t) \bfv_t + \dfrac{1}{dt} \E_t[d\bfv_t]\\
\dfrac{d\bfg_t}{dt} & = \bfA(\bfv_t;\bfp_t)^T\bfg_t\\
dZ_t & = (\mu_1 Z_t+\mu_2)\,dt + \sigma\, dW_t\\
\bfp_t & = \bfF(\bfg_t; Z_t).
\end{align*}
The first equation is just the HJB, where the RHS reflects the fact that technically, with Ito calculus, $d\bfv$ represents a stochastic integral, so ``technically'', you can't take the time derivative. The second line is the Fokker-Planck/KFE. Since we are not looking for a stationary density, the zeros become the time derivative of $\bfg$. The third equation is just some generic linear SDE, where $\mu_1,\mu_2$ are assumed to not depend on $\bfv,\bfg,Z$. Finally, the last equation is market-clearing.

We now re-arrange time derivatives onto the same side. Note $\bfp$ is statically solved, so technically we don't need to track it as a state. After plugging $\bfp$ directly into our equations, we have
\begin{align*}
 \E_t[d\bfv_t] & = \left[\rho \bfv_t - \bfu(\bfv_t) - \bfA(\bfv_t;\bfp_t) \bfv_t\right] \,dt\\
d\bfg_t & = \bfA(\bfv_t;\bfp_t)^T\bfg_t\,dt\\
dZ_t & = (\mu_1 Z_t+\mu_2)\,dt + \sigma\, dW_t
\end{align*}
We may re-write this nonlinear system as
\[d\bfs_t = \bff(\bfv_t, \bfg_t, p_t, Z_t), \]
with $d\bfs_t = 0$ at the steady-state. We now linearize the model by taking the first-order Taylor expansions w.r.t. $\bfv, \bfg, Z, \bfp$. Since all the arguments of $\bff$ are just dependent on these \textit{levels} of state variables, differentiation should not pop out any time derivatives/stochastic differentials. Let $d\hat{\bfg}_t$ represent the linear approximation of the time differential of the distribution (i.e. an approximation of $d\bfg_t$. Similar notation is used for the other state variables. Noting that $d\bfs_t = 0$ at the steady state, we can write, in gensys form,
\begin{align*}
\begin{bmatrix}
\E[d\hat{\bfv_t}]\\
d\hat{\bfg_t}\\
dZ_t
\end{bmatrix} & = \bfB\begin{bmatrix}\hat{\bfv_t}\\ \hat{\bfg_t}\\ Z_t + \sigma dW_t.
\end{bmatrix}
\end{align*}
In Sims's notation, we have
\[ \dot{s} = \Gamma_1 s + \Psi dW + \Pi \eta  \]
since we can completely summarize the time derivatives/differentials with their current levels at time $t$.




\end{document}
